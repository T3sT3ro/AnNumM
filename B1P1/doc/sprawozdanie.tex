\documentclass{article}
\usepackage[utf8]{inputenc}
\usepackage{polski}
\usepackage[polish]{babel}
\usepackage{graphicx}    
\usepackage{caption}
\usepackage{subcaption}
\usepackage{epstopdf}
\usepackage{amsmath}
\usepackage{amsthm}
\usepackage{hyperref}
\usepackage{url}
\usepackage{comment}
\usepackage{listings}
\newtheorem{defi}{Definicja}
\newtheorem{twr}{Twierdzenie}
\setlength{\parindent}{0pt}



\begin{document}
\title{Analiza numeryczna (M) - Pracownia 1 - Zadanie P1.9\\
Implementacja i analiza metody obliczania logarytmu sposobem Henrego Briggsa zaproponowanej w \cite{al-mohy11}\\}
\date{Październik 23, 2018}
\author{Maksymilian Polarczyk}
\maketitle

\section{Wstęp}

Awad H. Al-Mohy przedstawił w \cite{al-mohy11} udoskonaloną pod względem numerycznym metodę Briggsa obliczania logarytmu. Pierwotna metoda Henrego Briggsa użyta do przybliżania wartości logarytmów opiera się na własności logarytmu.

Program zaimplementowano z wykorzystaniem języka \textbf{Julia}, w pliku "program.jl". Wykresy zostały narysowane przy pomocy biblioteki \textbf{Plotly} w pliku "program.ipynb".



\section{Metoda Briggsa wyznaczania logarytmu}
\subsection{Oryginalna metoda}
\subsubsection{Wersja Awada H. Al-Mohy-ego}

\pagebreak

\section{Implementacja}
	\begin{lstlisting}
	k2 = k
	if arg(a) >= pi/2
		a = a^(1/2)
		k2 = k-1
	end
	z0 = a-1
	a = a^(1/2)
	r = 1 + a
	for j = 1:k2-1
		a = a^(1/2)
		r = r(1 + a)
	end
	r = z0 / r	
	\end{lstlisting}
\section{Przykłady}

	\subsubsection{subsection}
		\#TODO
		

\section{Wnioski}

	\#TODO
	
\begin{thebibliography}{9}
    
\bibitem{al-mohy11} Awad H. Al-Mohy,
\emph{A more accurate Briggs method for the logarithm},
Numerical Algorithms (2011), w druku,
DOI: 10.1007/s11075-011-9496-z.
    
\end{thebibliography}
\end{document}