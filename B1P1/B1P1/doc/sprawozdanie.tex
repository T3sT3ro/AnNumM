\documentclass{article}
\usepackage[utf8]{inputenc}
\usepackage{polski}
\usepackage[polish]{babel}
\usepackage{graphicx}    
\usepackage{caption}
\usepackage{subcaption}
\usepackage{epstopdf}
\usepackage{amsmath}
\usepackage{amsthm}
\usepackage{hyperref}
\usepackage{url}
\usepackage{comment}
\usepackage{listings}
\newtheorem{defi}{Definicja}
\newtheorem{twr}{Twierdzenie}
\setlength{\parindent}{0pt}



\begin{document}
\title{Analiza numeryczna (M) - Pracownia 1 - Zadanie P1.9\\
Implementacja i analiza metody obliczania logarytmu sposobem Henrego Briggsa zaproponowanej w \cite{al-mohy11}\\}
\date{Październik 23, 2018}
\author{Maksymilian Polarczyk}
\maketitle

\section{Wstęp}

Awad H. Al-Mohy przedstawił w \cite{al-mohy11} udoskonaloną pod względem numerycznym metodę Briggsa obliczania logarytmu. Pierwotna metoda Henrego Briggsa użyta do przybliżania wartości logarytmów opiera się na własności logarytmu.

Program zaimplementowano z wykorzystaniem języka \textbf{Julia}, w pliku "program.jl". Wykresy zostały narysowane przy pomocy biblioteki \textbf{Plots} w pliku "program.ipynb".



\section{Metoda Briggsa wyznaczania logarytmu}
\subsection{Oryginalna metoda}
	\#TODO 
\subsubsection{Wersja Awada H. Al-Mohy-ego}
	\#TODO
\pagebreak

\section{Uwarunkowanie}
Oryginalna metoda Briggsa jest podatna na utratę cyfr znaczących dla wartości $\sqrt[2^k]{a}$ bliskich zera (czyli efektywnie każdych wartości $\mathrm{a}$ przy większej liczbie iteracji) przez wyrażenie $\sqrt[2^k]{a} - 1$. Wersja Al-Mohy-ego korzystająca z wzoru skróconego mnożenia na różnicę kwadratów eliminuje powyższy problem przez zamianę różnicy na iloczyn sum liczb o części rzeczywistej większej niż $\mathrm{-1}$. 
	
	\#TODO matematyczny zapis
	
\section{Implementacja}
\subsection{Wersja Briggsa}
	\begin{lstlisting}
	briggs1(x, k):
		for i = 1:k
			a = a^(1/2)
		end
		r = a - 1
	\end{lstlisting}
\subsection{Wersja H. Al-Mohy}
	\begin{lstlisting}
	briggs2(x, k):
		k2 = k
		if arg(a) >= pi/2
			a = a^(1/2)
			k2 = k-1
		end
		z0 = a-1
		a = a^(1/2)
		r = 1 + a
		for j = 1:k2-1
			a = a^(1/2)
			r = r(1 + a)
		end
		r = z0 / r	
	\end{lstlisting}
\section{Eksperymenty}
	\subsection{eksperyment 1 - wartości krytyczne dla liczb zespolonych na małych, średnich i dużych zakresach}
	\#TODO - odnościk do jupytera eksperymentu 1.
	
	\subsection{eksperyment 2 - porównanie błędów względnych obu algorytmów dla różnych wartości $\mathrm{k}$}
	\#TODO - odnośnik do jupytera eksperymentu 2.
	
	\subsection{eksperyment 3 - monotoniczność wartości błędy względnego algorytmu 2 dla dużych $\mathrm{k}$ na $\Re$}
	\#TODO - odnośnik do jupytera eksperymentu 2.
	
\section{Wnioski}
	\#TODO
	
\begin{thebibliography}{9}
    
\bibitem{al-mohy11} Awad H. Al-Mohy,
\emph{A more accurate Briggs method for the logarithm},
Numerical Algorithms (2011), w druku,
DOI: 10.1007/s11075-011-9496-z.
    
\end{thebibliography}
\end{document}